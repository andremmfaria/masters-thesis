\chapter{Conclusion}
\label{sec:Conclusion}
\label{Chapter5} % For referencing the chapter elsewhere, use \ref{Chapter5} 

% Briefly restate the research problem and objectives.
% Remind the reader of the significance of your topic (LLMs + RAG for static malware analysis).
% Provide a roadmap of what this chapter will cover.
TODO...

\section{Summary of Research}
% Concisely summarize your overall research journey:
%   - Literature gaps you identified
%   - Your methodological approach
%   - Key components of your system (RAG pipeline, prompt engineering, static feature analysis)
% Keep this section high-level and focused on key ideas, not details.
TODO...

\section{Answers to Research Questions}
% Clearly address each of your research questions.
% Use a separate paragraph for each question:
%   - Q1: How can LLMs be leveraged to enhance static malware analysis?
%   - Q2: How does a RAG-enhanced LLM compare to traditional techniques?
%   - Q3: What role does RAG play in improving contextual relevance and interpretability?
% For each, summarize what your findings reveal and how they contribute new insights.
TODO...

\section{Research Contributions}
% Highlight your original contributions to the field.
% Examples:
%   - A novel framework integrating RAG with LLMs for malware classification
%   - Improved interpretability through context-aware prompting
%   - A reproducible evaluation pipeline for comparing static analysis tools with AI models
% Emphasize why these are valuable for academia or practice.
TODO...

\section{Implications of the Study}
% Discuss what your findings mean for the broader field of cybersecurity.
% Reflect on practical applications, such as:
%   - Security operations
%   - Threat analysis workflows
%   - Explainable AI in malware detection
% Include both theoretical and real-world implications.
TODO...

\section{Limitations}
% Acknowledge key limitations of your research.
% Examples:
%   - Focus limited to static analysis (not dynamic or hybrid)
%   - Limited dataset scope or quality of RAG source material
%   - Dependence on general-purpose LLMs with no fine-tuning
% Be honest and objective, but also note where the limitations can be addressed in future work.
TODO...

\section{Recommendations for Future Research}
% Suggest specific areas where future research could build on your work.
% Possibilities include:
%   - Incorporating dynamic features alongside static analysis
%   - Using fine-tuned LLMs or domain-specific models
%   - Evaluating RAG performance with higher-quality curated corpora
%   - Studying robustness against adversarial prompts or evasive malware
TODO...

\section{Final Reflections}
% End with a strong, reflective closing paragraph.
% Reiterate the importance of your research and what you’ve learned as a researcher.
% Reinforce your contribution to the field and the potential for your work to
% inform further innovation.
TODO...
