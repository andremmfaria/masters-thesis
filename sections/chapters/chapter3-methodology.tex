\chapter{Methodology}
\label{sec:Method}
\label{Chapter3} % For referencing the chapter elsewhere, use \ref{Chapter3}

% Introduce the purpose of this chapter.
% Reiterate your research aim: enhancing static malware analysis with LLMs and RAG.
% Provide a brief overview of the methodology, including data sources,
% architecture, evaluation strategy, and toolchain.
TODO...

\section{Research Design}
% Describe the nature of your research: exploratory, experimental, comparative.
% Explain the reasoning for using a prototype-based implementation and evaluation.
% Emphasize that your approach avoids LLM fine-tuning and instead relies on
% prompt engineering and external context via RAG.
TODO...

\section{System Architecture}
% Provide a high-level description of your system pipeline.
% Include a diagram (later) showing flow: malware sample → static feature extraction → retrieval → prompt generation → LLM response.
% List major system components and their interactions.
TODO...

\section{Data Sources}
% Describe where you obtain your malware samples and intelligence documents.
% Malware sources:
%   - MalwareBazaar for static binaries and metadata.
%   - VirusTotal (optional, via API) for reports and behavior data.
%     Detail any preprocessing steps: parsing PE headers, extracting bytecode,
%     API call strings, etc.
TODO...

\section{Retrieval-Augmented Generation (RAG) Pipeline}
% Explain how RAG is constructed using Haystack and pgvector.
% Describe:
%   - Embedding model used to vectorize documents.
%   - How documents (e.g., malware reports) are stored in PostgreSQL.
%   - How semantic retrieval is performed and injected into the LLM prompt.
%     Clarify how this improves contextual relevance and grounding in known
%     threat data.
TODO...

\section{Prompt Engineering}
% Discuss how you craft prompts for different tasks: classification, explanation, and comparison.
% Include examples of how static features and retrieved context are structured in the prompt.
% Explain the decision to use GPT-4 and/or DeepSeek-R1 in inference-only mode.
TODO...

\section{Toolchain}
This section outlines the used tooling for the experiments and some discussion
on why they were chosen.

\subsection{Programming Language and Environment}
% Python is used for implementation.
% PDM (Python Development Master) manages dependencies and virtual environments.
TODO...

\subsection{Language Models}
% GPT-4 (via OpenAI API) and/or DeepSeek-R1 are used for inference.
% Models are not fine-tuned; they rely on prompt engineering and external
% retrieval for task performance.
TODO...

\subsection{RAG Framework and Storage}
% Haystack is used to build the retrieval pipeline.
% PostgreSQL with pgvector is used to store and query vector embeddings of
% malware-related documents.
TODO...

\subsection{Malware Intelligence Sources}
% MalwareBazaar is used to obtain real-world malware samples and metadata.
% VirusTotal may be used (via API) to enrich data with threat reports and
% behavioral indicators.
TODO...

\subsection{Evaluation and Visualization}
% Scikit-learn is used to compute performance metrics (e.g., accuracy, F1 score).
% Pandas and Matplotlib are used for data handling and visualization.
% All experiments are version-controlled and reproducible via PDM.
TODO...

\section{Evaluation Framework}
% Define your evaluation metrics: accuracy, efficiency, interpretability.
% Describe baseline comparisons against traditional static malware classifiers (if applicable).
% Discuss any human-in-the-loop components for evaluating explanation quality.
TODO...

\section{Reliability, Validity, and Limitations}
% Discuss how reproducibility is ensured via version control, fixed seeds, and public datasets.
% Address limitations:
%   - Limited to static analysis.
%   - Dependent on retrieval quality and LLM reliability.
%   - No user-study or adversarial robustness testing (unless planned).
TODO...

\section{Ethical Considerations}
% Ensure ethical use of malware datasets.
% Describe safety protocols: no execution of live malware, all samples are statically analyzed.
% Highlight responsible use of LLMs and privacy-conscious data handling if using
% VirusTotal.
TODO...

\section{Summary}
% Recap the methodology structure and justify how it addresses your research questions.
% Set up the transition into the next chapter, where implementation details or
% results will follow.
TODO...
