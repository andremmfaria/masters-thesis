\chapter{Introduction}
\label{sec:Introduction}
\label{Chapter1} % For referencing the chapter elsewhere, use \ref{Chapter1} 

Ever since the creation of the first computer by Charles Babbage in the 19th century, computers
have been evolving at a rapid pace performing more and more tasks as the time passes. With this
evolution, malicious actors began to notice manners in which to subvert established systems. These
individuals, by force of belief or personal gain, have caused losses in the trillions to
organizations and individuals across the globe.

As a response to these threats, these organizations and individuals began to develop techniques and
tools to counter malicious actors. It has then, started the cat-and-mouse race between criminals
and agents of the law. One of these techniques revolves around the analysis of what software
adversaries develop to understand how they operate and the vulnerabilities they exploit to weaken
or outright topple established security systems.

This thesis presents a research about, and creation of, a method where RAG powered LLMs are
introduced onto to the malware static analysis workflow in order to increase efficiency on the
analysis.

\section{Background and Motivation}
% Introduce static malware analysis as a core method in cybersecurity.
% Explain the emergence of LLMs in code and text understanding.
% Define key concepts: static analysis, LLM, RAG, prompt engineering.
% Motivate the relevance: Why exploring LLMs for malware analysis is timely and important.
TODO...

\subsection{Static Malware Analysis in Cybersecurity}
% Define static malware analysis — Describe it as the process of examining a file's structure, code, or metadata without executing it, using techniques like disassembly, string analysis, and header inspection.
% Highlight its importance — Explain why static analysis is critical: it's safer than dynamic analysis, allows early detection, and can be automated for large-scale scanning (e.g., in antivirus engines or intrusion detection systems).
% Mention its limitations — Bring up common challenges like obfuscation, polymorphism, and difficulty in interpreting results without human expertise.
Static malware analysis allows analysts to scrutinize a file's bytecode, strings, imported
functions, and file headers to uncover malicious intent. This tried-and-true method of inspecting a
program's structure, instructions, and information without actually running it, is still one of the
most popular and dependable tools in the cybersecurity toolbox. It is essential in settings where
safety, speed, and scalability are critical since it eliminates the possibility of running
potentially dangerous code. It is the keystone in early malware identification and categorization,
due to its automation capabilities (i.e. antivirus engines or intrusion detection systems).

However, as threat actors evolve their tactics and adopt increasingly sophisticated evasion
techniques, static analysis has begun to show its limitations. Many contemporary malware variants
leverage obfuscation, packing, encryption, and code polymorphism to mask their functionality and
frustrate traditional static inspection. Moreover, extracting actionable intelligence from static
features often requires expert-level knowledge in assembly language, file formats, and API
behavior—creating a steep barrier for automation. These challenges have shown the need for
intelligent, context-aware tools capable of shortening the gap between low-level code patterns and
high-level semantic understanding.

\subsection{Emergence of LLMs in Code and Text Understanding}
% Briefly introduce LLMs — Explain what Large Language Models are: deep learning models trained on massive corpora of text, capable of understanding and generating natural language and code.
% Show their relevance to cybersecurity — Point out how LLMs are increasingly used in security applications, such as vulnerability detection, phishing detection, code summarization, and reverse engineering support.
% Discuss LLMs' strength in reasoning and pattern recognition — Emphasize how LLMs can understand syntactic and semantic patterns in both code and text, making them well-suited to analyze features like API call patterns or embedded strings in malware.
% Note the research opportunity — Highlight that while LLMs have been explored for general code analysis and some security tasks, their use in static malware analysis — particularly for classification and explanation — remains underexplored.
Over the past few years, Large Language Models (LLMs) have rapidly transitioned from experimental
research tools to foundational components across a wide range of natural language processing and
code analysis applications. Built on transformer-based architectures and trained on vast corpora
comprising not only natural language text but also programming languages and technical
documentation, these models have demonstrated an impressive capacity to generate, summarize,
translate, and reason about code and human language alike. Their success in handling structured and
semi-structured data has opened new avenues in fields such as automated code completion, bug
detection, documentation generation, and even code synthesis. With models like GPT, CodeBERT, and
StarCoder pushing the limits of scale and capability, LLMs are starting to play a major part in
determining how we interact with intricate codebases and interpret syntactic patterns that once
required domain-specific knowledge.

This increased capacity in both textual and programmatic reasoning has generated interest in
bringing LLMs to cybersecurity, where duties such as code interpretation, vulnerability
identification, and malware classification have historically been left for highly trained analysts.
Unlike rule-based systems, which rely on predetermined signatures or static heuristics, LLMs may
generalise from a variety of examples and adapt to novel or obfuscated code patterns, which is very
valuable in malware research. Furthermore, their capacity for natural language explanation enables
a level of interpretability that aligns well with security workflows that demand transparency and
traceability. While LLMs have shown early promise in tasks such as vulnerability detection and
reverse engineering assistance, their full potential in enhancing static malware
analysis, especially through integration with contextual retrieval systems, remains a largely
untapped area of research, presenting an opportunity to reimagine traditional security paradigms
with the help of advanced, language-aware intelligence.

\subsection{Motivation}
The reasoning behind the focus of this research is that static analysis remains to be a critical
component of malware detection and neutralization. Security analysts are often required to analyze
and process huge amounts of obfuscated data. By integrating RAG powered LLMs to this process we can
make the analysis easier by tapping onto contextual knowledge from the RAG and the analysis power
of LLMs.

\subsection{Key Concepts}
Here are defined all of the concepts and/or tools that are mentioned.

\subsubsection{Large Language Models}
Large Language Models (LLMs) are a type of multi-parameter Machine Learning (ML) algorithm designed
for natural language processing. These models that are trained on a large volume of text using a
multitude of learning techniques such as supervised and self-supervised learning.

Generative pre-trained transformers (GPTs) are the biggest and most powerful LLMs. Current models
can be adapted to certain tasks (via fine-tuning) or directed by prompt engineering
	[\cite{brown2020languagemodelsfewshotlearners}]. These models learn to predict the syntax,
semantics and ontologies found in human language, but they also pick up biases and errors from the
training data [\cite{10.1162/daed_a_01905}]. Popular examples of LLMs include:
\begin{itemize}
	\item GPT models (like GPT-4 or ChatGPT)
	\item Google's PaLM and Gemini
	\item Meta Llama series models
\end{itemize}

\subsubsection{Prompt Engineering}
Interactions with LLMs are made through \textit{prompts}. Those are natural language text segments
that describe the task the AI should perform [\cite{radford2019language}]. They can be queries,
commands or statements including context, instructions and/or relevant information.

\textit{Prompt engineering} is the action of manipulating these text segments in order to optimize
the wanted output [\cite{wahle-etal-2024-paraphrase}]. For example, providing a prompt such as "What is an
apple?" we have this output:\\

\fbox{%
	\begin{minipage}{0.9\linewidth}
		An apple is a round, edible fruit produced by the apple tree (Malus domestica). It typically
		has smooth skin that can be red, green, or yellow, and a sweet to tart taste depending on
		the variety. Apples have a firm white flesh and contain a core with small black or brown
		seeds.

		They're commonly eaten raw, but also used in cooking and baking-like in apple pies, sauces, and
		juices. Apples are rich in fiber, vitamin C, and antioxidants, making them a popular choice for a
		healthy snack. The saying "an apple a day keeps the doctor away" hints at their reputation for
		promoting good health.

		Want to know about their origin, symbolism, or maybe how they're grown?
	\end{minipage}
}\\

This output is acceptable, however it can be modified with some prompt engineering, such as an ask
to change the output formatting. Then, providing the phrase "Provide only bullet points." in
addition ot the original we have:\\

\fbox{%
	\begin{minipage}{0.9\linewidth}
		* Edible fruit from the apple tree (Malus domestica)\\
		* Typically round with red, green, or yellow skin\\
		* White, crisp flesh with a sweet to tart flavor\\
		* Contains a core with small seeds\\
		* Commonly eaten raw or used in cooking and baking\\
		* Rich in fiber, vitamin C, and antioxidants\\
		* Grown in temperate climates worldwide\\
		* Symbol of health, knowledge, and temptation in various cultures
	\end{minipage}
}\\

\subsubsection{Retrieval-Augmented Generation}
Retrieval-Augmented Generation (RAG)
[\cite{lewis2021retrievalaugmentedgenerationknowledgeintensivenlp}] is a technique that allows for
LLMs to query information from a specific type of datastore and use it as context for the response
it provides. This allows for adaptive learning of the models without the need of retraining
(fine-tuning). It improves the quality of LLM output as the model does not rely solely on
pre-trained knowledge but also on relevant information from the datastore.

RAG works by having the LLM to query the datastore using vector search, embeddings or keyword
matching (depending on the datastore type), feeding this data onto the LLM input and then
continuing with the normal LLM process for output generation.

\subsubsection{Malware}
Short for \textit{Malicious Software}, Malware is any program or code that is created for malicious
purposes like exploitation of computer systems networks or users. Often created with monetary gain
in mind, these programs activities include stealing sensitive data, damage systems, disrupt
operations or gain unauthorized access to systems or environments. Common types of malware include:
\begin{itemize}
	\item Viruses: Attach themselves to legitimate software, spreading when the software is run.
	\item Worms: Autonomous malware capable of multiplying across networks without needing human
	      intervention. Exploit network weaknesses to infiltrate other systems without permission.
	\item Trojans: Also known as Trojan Horses, these are disguised as genuine software to trick users into
	      running them.
	\item Ransomware: Encrypts or locks files, demanding payment (ransom) for restoration.
	\item Spyware: Stealthily gathers sensitive data and user activities without the user’s consent.
	\item Adware: Delivers unwanted advertisements, potentially slowing down or compromising systems.
\end{itemize}

\subsubsection{Malware Analysis}
As mentioned before, the software created by malicious actors needs to be evaluated and analyzed to
understand how it works, what exploits it takes advantage of and how these exploits can be patched.
As a basis, there are two types of analysis of software that can be made to understand any kind of
software. These are as follows:
\begin{itemize}
	\item \textbf{Static Analysis}: Focuses on looking at the sequences of individual instructions on the software
	      bytecode to identify the its characteristics, such as identification of the parameters in which the
	      analyzed software is built upon (e.g. Operating system it executes on, processor architecture,
	      etc), what execution paths it employs, what system calls it performs, etc. This analysis is done
	      without running the code and it requires a lot of time and effort from the analyst.
	\item \textbf{Dynamic Analysis}: Entails running the potential malicious software to gather runtime information
	      such as network calls and system call execution data which are not visible through static analysis.
	      This analysis is done by creating a special (sandbox) environment (i.e docker container and/or a virtual
	      machine) and running the software.
\end{itemize}

\subsubsection{Static Analysis vs Dynamic Analysis}
Both static and dynamic analysis have their own benefits and are important on their own right for
understanding how malware is designed and developed. While static analysis takes advantage of its
speed and safety where files do not need to be executed to be analyzed, dynamic analysis enables a
more extensive analysis which might not be visible with static analysis alone.

However, both have their limitations such as the increased computational cost and requirement of a
secure environment of dynamic analysis versus the vulnerability of static analysis to obfuscation
and encryption techniques.

\section{Problem Statement}
% Identify limitations of traditional static malware analysis (e.g., limited interpretability, rule-based constraints).
% Mention the challenges faced when scaling or adapting to new threats.
% Explain how LLMs and RAG might address these issues—but that there is limited research on combining them effectively for malware analysis.
TODO...

\section{Research Aim and Objectives}
% Aim: Clearly state the overarching goal (e.g., to enhance static malware analysis using RAG-enhanced LLMs).
% Objectives (list in bullet form, e.g.):
%     Develop a method for integrating LLMs with RAG.
%     Evaluate effectiveness compared to traditional static analysis.
%     Identify benefits and risks such as hallucinations or adversarial evasion.
TODO...

\subsection{Aim}
Research the enhancement of static malware analysis using RAG-enhanced LLMs

\subsection{Objectives}
\begin{itemize}
	\item Develop a method for integrating LLMs with RAG to enhance static malware analysis.
	\item Evaluate the effectiveness of RAG-enhanced LLMs in identifying, explaining, and comparing malware
	      threats using static features (e.g., file structure, bytecode, API calls) against traditional
	      static malware analysis techniques in terms of accuracy, efficiency, and interpretability, while
	      identifying challenges and potential risks (e.g., adversarial manipulation and hallucination).
\end{itemize}

\section{Research Questions and Hypothesis}
% List your three refined research questions.
% Optionally, include a hypothesis for each (e.g., “A RAG-enhanced LLM will outperform traditional static analysis in interpretability.”)

\subsection{Objectives}
These are the research questions that will guide the development of this thesis.
\begin{itemize}
	\item How LLMs can be leveraged to enhance static malware analysis?
	\item How does a RAG-enhanced LLM compare to traditional static analysis techniques in terms of accuracy,
	      efficiency, and interpretability?
	\item What role does RAG play in improving LLM-driven malware classification, particularly in terms of
	      contextual relevance and justifiability of results?
\end{itemize}

\subsection{Hypothesis}
\begin{itemize}
	\item Hypothesis 1: LLMs can enhance static malware analysis by accurately interpreting and classifying
	      malware samples based on static features such as bytecode, file structure, and API calls, offering
	      greater adaptability and insight than rule-based static tools.
	\item Hypothesis 2: A RAG-enhanced LLM will outperform traditional static analysis techniques in terms of
	      interpretability and contextual understanding, while maintaining comparable levels of accuracy and
	      efficiency in malware classification.
	\item Hypothesis 3: RAG improves LLM-driven malware classification by providing relevant contextual
	      information during inference, leading to more justifiable and semantically grounded explanations of
	      classification results.
\end{itemize}

\section{Methodology Overview}
% Briefly describe the approach:
%     Focus on static features (like bytecode, API calls).
%     Use LLMs with prompt engineering instead of fine-tuning.
%     Retrieve contextual info through RAG to assist malware classification.
% Mention evaluation metrics (accuracy, interpretability, efficiency).
% Limitations or scope (e.g., no dynamic analysis, no expert consultation).
TODO...

\section{Contributions of the Research}
% Explain the potential impact or novelty of your research
% Clarify how it advances the field or addresses a specific gap
% Discuss how your work contributes:
%     A novel framework combining RAG + LLM for malware analysis.
%     Empirical comparison with traditional tools.
%     Insight into risks and limitations of AI-based approaches.
TODO...

\section{Thesis Structure}
% Provide a roadmap of the chapters in the thesis
% Refer to chapters by number and label (e.g., Chapter~\ref{sec:LitReview})
% Provide a chapter-by-chapter breakdown:
%     Chapter 2 (\ref{sec:LitReview}) reviews related work.
%     Chapter 3 (\ref{sec:Method}) details the methodology.
%     Chapter 4 presents results and discussion.
%     Chapter 5 concludes with key findings and future work.
TODO...
