\chapter{Literature Review}
\label{sec:LiteratureReview}
\label{Chapter2} % For referencing the chapter elsewhere, use \ref{Chapter2} 

% Briefly introduce the purpose of this chapter.
% Mention the key themes: static malware analysis, LLMs in cybersecurity, and RAG.
% Explain that the chapter identifies trends, evaluates contributions, and
% exposes gaps to justify your research.
TODO...

\section{Static Malware Analysis}
% Define static malware analysis and explain its significance.
% Discuss techniques such as:
%   - Signature-based detection
%   - Heuristic or rule-based analysis
%   - ML-based static classifiers
%     Highlight the limitations: poor generalization, obfuscation resistance,
%     lack of semantic context.
TODO...

\subsection{Signature-Based and Heuristic Methods}
% Describe traditional approaches and tools (e.g., YARA, ClamAV, AV engines).
% Mention why they’re fast but limited by pattern-dependence.
TODO...

\subsection{Machine Learning-Based Approaches}
% Explore the use of static features (e.g., API calls, bytecode) in ML models.
% Reference key datasets and benchmarks (e.g., EMBER, VirusShare).
% Highlight common issues: feature engineering overhead, poor explainability.
TODO...

\section{The Role of Language Models in Cybersecurity}
% Introduce LLMs (e.g., GPT, BERT, CodeBERT) in the context of cybersecurity.
% Highlight their ability to reason over code, extract threats, or assist in SOC workflows.
% Note both advantages (generalization, language reasoning) and challenges
% (hallucinations, black-box nature).
TODO...

\section{Retrieval-Augmented Generation (RAG) in NLP}
% Explain what RAG is: a combination of retrieval models and LLMs.
% Describe how it improves factuality and context-awareness in LLM outputs.
% Mention successful applications in NLP (e.g., QA, summarization).
% Start hinting at its potential benefits for malware analysis tasks.
TODO...

\section{Combining RAG and LLMs for Static Malware Analysis}
% Present any research, even conceptual, that combines RAG with malware classification.
% If the field is sparse, note how work in adjacent fields (e.g., RAG + code understanding) inspires your direction.
% Discuss why this integration can help with:
%   - Context-aware detection/classification
%   - Explainability of decisions
%   - Reducing hallucination by grounding in known malware samples/docs
TODO...

\section{Gaps in Current Literature}
% Identify clear gaps and connect them to your research questions.
% Examples:
%   - Few works integrate RAG with static analysis for malware
%   - LLM evaluations often ignore interpretability
%   - No benchmarking of RAG-LLM models vs traditional static tools
% Use this section to justify your proposed method and scope.
TODO...

\section{Summary}
% Recap key points and themes from the chapter.
% Mention how insights from the literature guide your upcoming methodology.
% Emphasize how your work addresses the identified gaps and pushes the field
% forward.
TODO...
