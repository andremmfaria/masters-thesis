%----------------------------------------------------------------------------------------
%	PACKAGES AND OTHER DOCUMENT CONFIGURATIONS
%----------------------------------------------------------------------------------------

\documentclass[
11pt, % The default document font size, options: 10pt, 11pt, 12pt
%oneside, % Two side (alternating margins) for binding by default, uncomment to switch to one side
english, % ngerman for German
singlespacing, % Single line spacing, alternatives: onehalfspacing or doublespacing
%draft, % Uncomment to enable draft mode (no pictures, no links, overfull hboxes indicated)
%nolistspacing, % If the document is onehalfspacing or doublespacing, uncomment this to set spacing in lists to single
%liststotoc, % Uncomment to add the list of figures/tables/etc to the table of contents
%toctotoc, % Uncomment to add the main table of contents to the table of contents
%parskip, % Uncomment to add space between paragraphs
%nohyperref, % Uncomment to not load the hyperref package
headsepline, % Uncomment to get a line under the header
%chapterinoneline, % Uncomment to place the chapter title next to the number on one line
%consistentlayout, % Uncomment to change the layout of the declaration, abstract and acknowledgements pages to match the default layout
]{style} % The class file specifying the document structure

\usepackage[utf8]{inputenc} % Required for inputting international characters
\usepackage[T1]{fontenc} % Output font encoding for international characters

\usepackage{mathpazo} % Use the Palatino font by default

\usepackage[backend=bibtex,style=authoryear,natbib=true]{biblatex} % Use the bibtex backend with the authoryear citation style (which resembles APA)

\addbibresource{./misc/bibliography.bib} % The filename of the bibliography

\usepackage[autostyle=true]{csquotes} % Required to generate language-dependent quotes in the bibliography

%----------------------------------------------------------------------------------------
%	MARGIN SETTINGS
%----------------------------------------------------------------------------------------

\geometry{
	paper=a4paper, % Change to letterpaper for US letter
	inner=2.5cm, % Inner margin
	outer=3.8cm, % Outer margin
	bindingoffset=.5cm, % Binding offset
	top=1.5cm, % Top margin
	bottom=1.5cm, % Bottom margin
	%showframe, % Uncomment to show how the type block is set on the page
}

%----------------------------------------------------------------------------------------
%	THESIS INFORMATION
%----------------------------------------------------------------------------------------

\thesistitle{Automated Malware Analysis Using Large Language Models} % Your thesis title, this is used in the title and abstract, print it elsewhere with \ttitle
\supervisor{Dr Robert G \textsc{Smith}} % Your supervisor's name, this is used in the title page, print it elsewhere with \supname
\examiner{} % Your examiner's name, this is not currently used anywhere in the template, print it elsewhere with \examname
\degree{M.Sc} % Your degree name, this is used in the title page and abstract, print it elsewhere with \degreename
\author{Andre \textsc{Faria}} % Your name, this is used in the title page and abstract, print it elsewhere with \authorname
\addresses{Dublin, Ireland} % Your address, this is not currently used anywhere in the template, print it elsewhere with \addressname

\subject{Cybersecurity} % Your subject area, this is not currently used anywhere in the template, print it elsewhere with \subjectname
\keywords{} % Keywords for your thesis, this is not currently used anywhere in the template, print it elsewhere with \keywordnames
\university{\href{https://www.tudublin.ie/}{Technological University Dublin}} % Your university's name and URL, this is used in the title page and abstract, print it elsewhere with \univname
\department{\href{https://www.tudublin.ie/explore/faculties-and-schools/computing-digital-data/informatics-and-cybersecurity/}{School of Informatics and Cyber Security}} % Your department's name and URL, this is used in the title page and abstract, print it elsewhere with \deptname
\submitdate{May 2025} % The month and year that you submit your FINAL draft to the university, this is not currently used anywhere in the template, print it elsewhere with \subdate

\AtBeginDocument{
\hypersetup{pdftitle=\ttitle} % Set the PDF's title to your title
\hypersetup{pdfauthor=\authorname} % Set the PDF's author to your name
\hypersetup{pdfkeywords=\keywordnames} % Set the PDF's keywords to your keywords
}

\begin{document}

\frontmatter % Use roman page numbering style (i, ii, iii, iv...) for the pre-content pages

\pagestyle{plain} % Default to the plain heading style until the thesis style is called for the body content

%----------------------------------------------------------------------------------------
%	TITLE PAGE
%----------------------------------------------------------------------------------------

\begin{titlepage}
	\begin{center}

		\vspace*{.06\textheight}
		{\scshape\LARGE \univname\par}\vspace{1.5cm} % University name
		\textsc{\Large Masters Thesis}\\[0.5cm] % Thesis type

		\HRule \\[0.4cm] % Horizontal line
		{\huge \bfseries \ttitle\par}\vspace{0.4cm} % Thesis title
		\HRule \\[1.5cm] % Horizontal line

		\begin{minipage}[t]{0.4\textwidth}
			\begin{flushleft} \large
				\emph{Author:}\\
				\href{https://www.linkedin.com/in/andremmfaria/}{\authorname} % Author name - remove the \href bracket to remove the link
			\end{flushleft}
		\end{minipage}
		\begin{minipage}[t]{0.4\textwidth}
			\begin{flushright} \large
				\emph{Supervisor:} \\
				\href{https://www.tudublin.ie/explore/faculties-and-schools/computing-digital-data/informatics-and-cybersecurity/people/academic-staff/robertsmith.php}{\supname} % Supervisor name - remove the \href bracket to remove the link
			\end{flushright}
		\end{minipage}\\[2cm]

		\vfill

		\large \textit{A thesis submitted in fulfillment of the requirements\\ for the degree of \degreename}\\[0.3cm] % University requirement text
		\textit{in the}\\[0.4cm]
		\deptname\\[2cm] % Research group name and department name

		\includegraphics[width=0.2\textwidth]{./image/uni-logo.png} % University/department logo - uncomment to place it

		\vfill

		{\large \subdate}\\[4cm] % Date

		\vfill
	\end{center}
\end{titlepage}

%----------------------------------------------------------------------------------------
%	DECLARATION PAGE
%----------------------------------------------------------------------------------------

\begin{declaration}
	\addchaptertocentry{\authorshipname} % Add the declaration to the table of contents
	\noindent I, \authorname, declare that this thesis titled, \enquote{\ttitle} and the work presented in it are my own. I confirm that:

	\begin{itemize}
		\item This work was done wholly or mainly while in candidature for a research degree
		      at this Institute of Technology Blanchardstown.
		\item Where any part of this thesis has previously been submitted for a degree or any
		      other qualification at this University or any other institution, this has been
		      clearly stated.
		\item Where I have consulted the published work of others, this is always clearly
		      attributed.
		\item Where I have quoted from the work of others, the source is always given. With
		      the exception of such quotations, this thesis is entirely my own work.
		\item I have acknowledged all main sources of help.
		\item Where the thesis is based on work done by myself jointly with others, I have
		      made clear exactly what was done by others and what I have contributed
		      myself.\\
	\end{itemize}

	\noindent Signed:\\
	\rule[0.5em]{25em}{0.5pt} % This prints a line for the signature

	\noindent Date: \today \\
	\rule[0.5em]{25em}{0.5pt} % This prints a line to write the date
\end{declaration}

\cleardoublepage

%----------------------------------------------------------------------------------------
%	QUOTATION PAGE
%----------------------------------------------------------------------------------------

\vspace*{0.2\textheight}

\noindent\enquote{\itshape The true thesis are the friends we make along the way.}\bigbreak

\hfill Anonymous

%----------------------------------------------------------------------------------------
%	ABSTRACT PAGE
%----------------------------------------------------------------------------------------

\begin{abstract}
	\addchaptertocentry{\abstractname} % Add the abstract to the table of contents
	Despite the fact that an abstract is quite brief, it must do almost as much
	work as the multi-page paper that follows it. In a computer science paper, this
	means that it should in most cases include the following sections. Each section
	is typically a single sentence, although there is room for creativity. In
	particular, the parts may be merged or spread among a set of sentences. Use the
	following as a checklist for your next abstract (URL:
	http://www.ece.cmu.edu/~koopman/essays/abstract.html):

	\begin{description}
	\item[Motivation:] Why do we care about the problem and the results? If the problem
			isn't obviously "interesting" it might be better to put motivation first; but
			if your work is incremental progress on a problem that is widely recognized as
			important, then it is probably better to put the problem statement first to
			indicate which piece of the larger problem you are breaking off to work on.
			This section should include the importance of your work, the difficulty of the
			area, and the impact it might have if successful.
	\item[Problem statement:] What problem are you trying to solve? What is the scope of
			your work (a generalized approach, or for a specific situation)? Be careful not
			to use too much jargon. In some cases it is appropriate to put the problem
			statement before the motivation, but usually this only works if most readers
			already understand why the problem is important.
	\item[Approach:] How did you go about solving or making progress on the problem? Did
			you use simulation, analytic models, prototype construction, or analysis of
			field data for an actual product? What was the extent of your work (did you
			look at one application program or a hundred programs in twenty different
			programming languages?) What important variables did you control, ignore, or
			measure?
	\item[Results:] What's the answer? Specifically, most good computer architecture
			papers conclude that something is so many percent faster, cheaper, smaller, or
			otherwise better than something else. Put the result there, in numbers. Avoid
			vague, hand-waving results such as "very", "small", or "significant." If you
			must be vague, you are only given license to do so when you can talk about
			orders-of-magnitude improvement. There is a tension here in that you should not
			provide numbers that can be easily misinterpreted, but on the other hand you
			don't have room for all the caveats.
	\item[Conclusions:] What are the implications of your answer? Is it going to change
			the world (unlikely), be a significant "win", be a nice hack, or simply serve
			as a road sign indicating that this path is a waste of time (all of the
			previous results are useful). Are your results general, potentially
			generalizable, or specific to a particular case?

	\end{description}
\end{abstract}

%----------------------------------------------------------------------------------------
%	ACKNOWLEDGEMENTS
%----------------------------------------------------------------------------------------

\begin{acknowledgements}
	\addchaptertocentry{\acknowledgementname} % Add the acknowledgements to the table of contents
	The acknowledgments and the people to thank go here, don't forget to include your project advisor\ldots
\end{acknowledgements}

%----------------------------------------------------------------------------------------
%	LIST OF CONTENTS/FIGURES/TABLES PAGES
%----------------------------------------------------------------------------------------

\tableofcontents % Prints the main table of contents

\listoffigures % Prints the list of figures

\listoftables % Prints the list of tables

%----------------------------------------------------------------------------------------
%	ABBREVIATIONS
%----------------------------------------------------------------------------------------

\begin{abbreviations}{ll} % Include a list of abbreviations (a table of two columns)

	\textbf{LAH} & \textbf{L}ist \textbf{A}bbreviations \textbf{H}ere\\
	\textbf{WSF} & \textbf{W}hat (it) \textbf{S}tands \textbf{F}or\\

\end{abbreviations}

%----------------------------------------------------------------------------------------
%	PHYSICAL CONSTANTS/OTHER DEFINITIONS
%----------------------------------------------------------------------------------------

\begin{constants}{lr@{${}={}$}l} % The list of physical constants is a three column table

	% The \SI{}{} command is provided by the siunitx package, see its documentation for instructions on how to use it

	Speed of Light & $c_{0}$ & \SI{2.99792458e8}{\meter\per\second} (exact)\\
	%Constant Name & $Symbol$ & $Constant Value$ with units\\

\end{constants}

%----------------------------------------------------------------------------------------
%	SYMBOLS
%----------------------------------------------------------------------------------------

\begin{symbols}{lll} % Include a list of Symbols (a three column table)

	$a$ & distance & \si{\meter} \\
	$P$ & power & \si{\watt} (\si{\joule\per\second}) \\
	%Symbol & Name & Unit \\

	\addlinespace % Gap to separate the Roman symbols from the Greek

	$\omega$ & angular frequency & \si{\radian} \\

\end{symbols}

%----------------------------------------------------------------------------------------
%	DEDICATION
%----------------------------------------------------------------------------------------

\dedicatory{For/Dedicated to/To my\ldots}

%----------------------------------------------------------------------------------------
%	THESIS CONTENT - CHAPTERS
%----------------------------------------------------------------------------------------

\mainmatter % Begin numeric (1,2,3...) page numbering

\pagestyle{thesis} % Return the page headers back to the "thesis" style

% Include the chapters of the thesis as separate files from the Chapters folder
% Uncomment the lines as you write the chapters

\include{./chapter/Chapter1}
%\include{./chapter/Chapter2} 
%\include{./chapter/Chapter3}
%\include{./chapter/Chapter4} 
%\include{./chapter/Chapter5} 

%----------------------------------------------------------------------------------------
%	THESIS CONTENT - APPENDICES
%----------------------------------------------------------------------------------------

\appendix % Cue to tell LaTeX that the following "chapters" are Appendices

% Include the appendices of the thesis as separate files from the Appendices folder
% Uncomment the lines as you write the Appendices

\include{./appendice/AppendixA}
%\include{./appendice/AppendixB}
%\include{./appendice/AppendixC}

%----------------------------------------------------------------------------------------
%	BIBLIOGRAPHY
%----------------------------------------------------------------------------------------

\printbibliography[heading=bibintoc]

%----------------------------------------------------------------------------------------

\end{document}
